\documentclass[11pt, a4paper, ngerman]{article}

% IMPORTS
\usepackage{ntheorem}
\usepackage[ngerman]{babel}
\usepackage[utf8x]{inputenc}
\usepackage{amsfonts}
\usepackage{amsmath}
\usepackage{amssymb}
\usepackage{amstext}
\usepackage{mathtools} % Pfeile mit Beschriftung
%\usepackage{amsthm}

% NEWCOMMANDS
% Syntax: \newcommand{\shortcut}{\wasrauskommensoll}

\newcommand{\N}{\mathbb{N}}
\newcommand{\R}{\mathbb{R}}
\newcommand{\Z}{\mathbb{Z}}

% THEOREMSTYLES
\theorembodyfont{\upshape}
\newtheorem*{def*}{Definition}
\newtheorem{theorem}{Satz}
\newtheorem{aufgabe}{Aufgabe}

\begin{document}

\begin{aufgabe} (Mengensysteme)
    \begin{enumerate}
        \item 
        Ist der Schnitt zweier Semialgebren wieder eine Semialgebra ? 
        \item 
        Sei $\mathcal{A}$ eine Semialgebra. Geben Sie $\alpha(\mathcal{A})$ an. 
        \item 
        Sei $\mathcal{E} \subseteq \mathcal{P}(\Omega)$ beliebig. Geben Sie $\alpha(\mathcal{E})$ an.
        \item 
        Erläutern Sie die Beweistechnik des Good-Set-Principles anhand eines selbstgewählten Beispiels.
        \item 
        Skizzieren Sie kurz den Zusammenhang zwischen den folgenden Begriffen: Semialgebra, Algebra, $\sigma$-Algebra, Dynkin-System, monotones System. 
        \item 
        Wieso benötigen wir so viele verschiedene Arten von Mengesystemen? 
        \item 
        Was ist eine Borel-$\sigma$-Algebra? 
        \item 
        Geben Sie einen Erzeuger der Borelschen $\sigma$-Algebra im $\R^k$ an und skizzieren Sie den Beweis. 
    \end{enumerate}
\end{aufgabe}

\begin{aufgabe} (Messbare Funktionen)
    \begin{enumerate}
        \item
        Wann ist eine Abbildung messbar ? 
        \item 
        Unter welcher Voraussetzung kann man diese Bedingung ggf. "abschwächen"? 
        \item
        Was ist eine initiale $\sigma$-Algebra ?
        \item 
        Was besagt das Faktorisierungslemma ? 
        \item 
        Skizzieren Sie das allgemeine Vorgehen bei einem Beweis per algebraischer Induktion 
    \end{enumerate}
    
\end{aufgabe}

\begin{aufgabe} (Produkträume)
    \begin{enumerate}
        \item
        Definieren Sie das kartesische Produkt von Mengen $\Omega_i$, $i \in I \neq \emptyset$. 
        \item
        Wie ist das Produkt von $\sigma$-Algebren definiert?
        \item 
        Nennen Sie einen durchschnittstabilen Erzeuger der Produkt-$\sigma$-Algebra. 
        \item 
        Betrachte die Mengen 
        \begin{align*}
            A_1 &:= \{f:[0,1] \to \R \ \rvert \sup_{x \in [0,1]} \lvert f(x) \rvert < C \ \}, C \in (0,\infty) \\\
            A_2 &:= \{f:[0,1] \to \R \ \rvert \text{ f besitzt keine Nullstelle} \ \} \\\
            A_3 &:= \{f:[0,1] \to \R \ \rvert \text{ f ist stetig} \ \}. 
        \end{align*}
        Sind diese Mengen in $\bigotimes_{t \in [0,1]} \mathcal{B}$ enthalten? Begründen Sie Ihre Entscheidung. 

            

    \end{enumerate}
\end{aufgabe}

\begin{aufgabe} (Konstruktion von Maßen)
   \begin{enumerate}
        \item 
        Was ist der Unterschied zwischen einem Inhalt und einem Prämaß ? Was unterscheidet ein Maß von einem Prämaß?
        \item 
        Sei $\mu$ ein endlicher Inhalt. Geben sie vier zur folgenden Aussage äquivalente Aussagen an: $\mu$ ist $\sigma$-sub-additiv. 
        \item 
        Wie lässt sich ein auf einer Algebra $\mathcal{A}$ definiertes Prämaß $\mu$ zu einem Maß auf $\sigma(\mathcal{A})$ fortsetzen?
        \item 
        Welche Rolle spielt die  $/sigma$-Endlichkeit bei der Fortsetzung von Prämaßen?
        \item 
        Sei das Maß $\nu$ absolutstetig bzgl. dem Maß $\mu$. Besitzt $\nu$ dann eine $\mu$-Dichte?
        (Gegenbeispiel + zusätzliche Bedingung)
        \item 
        Was ist die Vervollständigung eines Maßraums?
        \item 
        In welchem Kontext benötigt man den Begriff der kompakten Approximierbarkeit?
        \item 
        Skizzieren Sie die Konstruktion des k-dimensionalen Lebesgue-Maßes. 
        
   \end{enumerate}
\end{aufgabe}

\begin{aufgabe} (Maßintegral \& fast-überall Eigenschaften)
    \begin{enumerate}
        \item 
        Definieren Sie das Maßintegral für eine messbare Funktion 
        \item
        Skizzieren sie die Beweisidee des Satzes von Beppo Levi
        \item 
        Was versteht man unter einer "fast überall" Eigenschaft ?
        \item 
        Seien $f,g \in \mathcal{F}(\Omega, \mathcal{A})$. Geben Sie eine möglichst schwache Bedingung dafür an, dass $f \leq g$ fast überall.
        \item 
        Was besagt der Satz von Radon-Nikodym? Wie sieht es bei nicht $\sigma$-endlichen Maßen aus? Gegenbeispiel?
        \item 
        Was ist eine $\sigma$-additive-Mengenfunktion und was versteht man unter einer Jordan-Zerlegung? Ist diese eindeutig? Begründen Sie Ihre Antwort. 
        \item 
        Was versteht man unter einer Hahn-Zerlegung?
        \item 
        Wann ist ein Maß singulär bzgl. einem anderen Maß ? 
        \item 
        Was ist die Lebesgue-Zerlegung eines Maßes ?
        \item 
        Sei $(\Omega, \mathcal{A})$ messbarer Raum und seien $\mu_1, \mu_2$ Wahrscheinlichkeitsmaße auf $\mathcal{A}$. Begründen Sie, dass dann auch $\mu := \frac{1}{2}(\mu_1 + \mu_2)$
        ein Wahrscheinlichkeitsmaß auf $\mathcal{A}$ ist und zeigen Sie, dass $\mu_i$ absolutstetig bzgl. $\mu$ ist. 
        \item 
        Sei $\mu$ das Zählmaß auf $\mathcal{P}(\N)$. Zeigen Sie, dass die Abbildung
        \begin{align*}
            f: \N \to \R, n \mapsto \frac{(-1)^n}{n}
        \end{align*}
        nicht $\mu$-integrierbar ist. 

    
    \end{enumerate}
\end{aufgabe}

\begin{aufgabe} (Maßkerne \& Produktmaße)
    \begin{enumerate}
        \item 
        Was ist ein Maßkern ? Welche Eigenschaften von Maßkernen haben wir in der Vorlesung kennengelernt ?
        \item
        Wie ist das Produktmaß eines $\sigma$-endlichen Maßes $\mu$ mit einem  $\sigma$-endlichen Maßkern K definiert. Welche charakterisierende Eigenschaft besitzt es?
        Wie zeigt man die Eindeutigkeit ? 
        \item 
        Was versteht man unter der Standardfortsetzung einer Funktion? Wofür wird sie gebraucht?
        \item 
        Formulieren sie den Satz von Fubini für Maßkerne.  
        \item 
        Erläutern Sie ein Anwendungsbeispiel des Satzes von Ionescu-Tulcea. 
        \item 
        Inwiefern kann der Satz von Cavalieri bei der Berechnung von Erwartungswerten behilflich sein? 
    \end{enumerate}
\end{aufgabe}

\begin{aufgabe} (Verteilungen \& Verteilungsfunktionen)
    \begin{enumerate}
        \item 
        Nennen Sie die charakterisierenden Eigenschaften einer Verteilungsfunktion.
        \item 
        Formulieren sie den Transformationssatz für Bildmaße. 
        \item 
        Was ist überhaupt ein Bildmaß ?
        \item 
        Wann besitzen zwei messbare Abbildungen das gleiche Bildmaß ? Gilt auch die Umkehrung? 
    \end{enumerate}
\end{aufgabe}

\begin{aufgabe} (Fast sicher, stochastische \& Verteilungskonvergenz)
    \begin{enumerate}
        \item 
        Definieren Sie fast sichere Konvergenz und geben Sie hinreichende und notwendige Bedingungen an. 
        \item 
        Formulieren Sie die Cauchy-Kriterien für fast-sichere und für stochastische Konvergenz. 
        \item
        Was besagt das Lemma von Pratt? Wo wird es verwendet?  
        \item 
        Welche Konvergenzform ist die "stärkste"? Welche Implikationen gelten? 
        \item 
        Geben Sie ein Beispiel dafür an, dass aus stochastischer Konvergenz im Allgemeinen nicht auch fast sichere Konvergenz folgt. 
        \item 
        Was können Sie über die Eindeutigkeit fast-sicherer/stochastischer Grenzwerte sagen? Wie schaut es bei Verteilungskonvergenz aus?
        \item
        Charakterisieren Sie die Konvergenz in Verteilung. 
        \item 
        Was besagt der Satz von Skorohod? 
        \item 
        Was ist stochastische Äquivalenz ?
    \end{enumerate}
\end{aufgabe}

\begin{aufgabe} (Konvergenz im p-ten Mittel \& gleichgradige Integrierbarkeit)
    \begin{enumerate}
        \item 
        Was ist die Minkowski-Ungleichung ? Wozu wird Sie verwendet?
        \item 
        Worin unterscheiden sich die Vektorräume $L_p$ und $\mathcal{L}_p$ ? 
        \item 
        Wieso betrachten wir ausschließlich $p \geq 1$? 
        \item 
        Unter welcher zusätzlichen Voraussetzung impliziert fast sichere Konvergenz die Konvergenz im (ersten) Mittel ($L_1$-Konvergenz)?
        \item 
        Definieren Sie gleichgradige Integrierbarkeit und geben Sie eine Charakterisierung an. 
        \item 
        Unter welcher zusätzlichen Bedingung folgt aus stochastischer Konvergenz auch die Konvergenz im p-ten Mittel?


    \end{enumerate}
\end{aufgabe}

\begin{aufgabe} (Unabhängigkeit \& 0-1-Gesetze)
   \begin{enumerate}
        \item 
        Definieren Sie die Unabhängigkeit von Zufallsvariablen.
        \item 
        Geben Sie eine möglichst schwache Bedingung für die Unabhängigkeit zweier Zufallsvariablen an. 
        \item 
        Zeigen oder widerlegen Sie: Sind $X,Y$ Zufallsgrößen mit  $P^{(X,Y)} = P^X \times P^Y$, dann sind $X,Y$ unabhängig. 
        \item 
        Wann sind zwei Zufallsvariablen $X,Y$ unkorreliert? Sind $X,Y$ dann auch unabhängig? 
        \item 
        Welche Modellierung bietet sich meist an, um eine Folge unabhängiger Zufallsgrößen mit gegebener Verteilung zu modellieren? 
        \item 
        Definieren Sie die Faltung zweier Wahrscheinlichkeitsmaße. 
        \item   
        Formulieren Sie das Lemma von Borel-Cantelli 
        \item 
        Skizzieren Sie die Beweisidee des 0-1-Gesetzes. 
        \item 
        Sei $f: \Omega \to \bar{\R}$ eine $\mathcal{A}_{\infty}$/$\bar{\mathcal{B}}$-messbare Abbildung. Was gilt dann fast sicher? 
    \end{enumerate}
    
\end{aufgabe}

\begin{aufgabe} (Gesetze der großen Zahlen)
    \begin{enumerate}
        \item 
        Nennen Sie hinreichende und notwendige Kriterien für die Konvergenz zufälliger Reihen. 
        \item 
        Sei $(X_n)_{n \in \N}$ eine iid Folge von Zufallsgrößen und $(c_i)_{i \in \N}$ eine Folge in $\R$. 
        Nennen Sie eine notwendige und hinreichende Bedingung für die Konvergenz der zufälligen Reihe $\sum_{n=1}^{\infty} c_n X_n$ .  
        \item 
        Formulieren sie zwei Versionen des Starken Gesetz der Großen Zahlen (SLLN). 
    \end{enumerate} 
\end{aufgabe}

\end{document}
